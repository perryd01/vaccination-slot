\newcommand{\function}[4]{
  \textbf{#1}%
  (\emph{#2}):
  #3 \newline
  \emph{\color{teal} #4}
}

\newcommand{\gopkg}[2]{
  \href{https://pkg.go.dev/github.com/perryd01/vaccination-slot/chaincode#1}{#2}%
}

\newcommand{\testStruct}[3]{
  \subsubsection{#1}
  \paragraph*{Steps}
  \begin{enumerate}
    #2
  \end{enumerate}
  \textbf{Expected outcome:} #3
}

\newcommand{\vsType}[0]{\gopkg{\#VaccinationType}{Type}}
\newcommand{\vsDate}[0]{\gopkg{\#VaccinationDate}{Date}}


\subsection{Design decisions}
\emph{Our interpretation of the task.} There are doctors and patients. The doctors are able to mint Vaccination Slot tokens. The doctors are able to burn \emph{(invalidate)} a token. The most important properties of a single token are:%
\vsType \emph{(of the vaccine)},
\vsDate \emph{(when the token should be burned)},
\href{https://pkg.go.dev/github.com/perryd01/vaccination-slot/chaincode\#VaccinationDate}{Burned} \emph{(is it used up)},
\href{https://pkg.go.dev/github.com/perryd01/vaccination-slot/chaincode\#VaccinationSlotData}{Previous} \emph{(previous vaccine type the patient got)}.

A patient can hold only one non-burned token and unlimited number of burned. A patient cannot trade a burned token.

\subsubsection{Trading Offers}
A patient can send an offer to another patient about trading a valid token for another. The offer can be accepted or declined. If the necessary conditions are available, the trade will be successful.
The offer will result in an error if any of the participants doesn't own the token mentioned in the offer or trying to trade burned tokens.

\subsection{Data model}
TokenId is generated as a Universal Unique Identifier. Type is actually an \emph{enum} and it can be any of the following:
\begin{itemize}
  \item Alpha
  \item Bravo
  \item Charlie
  \item Delta
  \item Echo
\end{itemize}

\begin{center}
  \begin{table}[!ht]
    \centering
    \begin{tabular}{|c | c | c | c |}
      \hline
      \multicolumn{4}{|c|}{Token}                                              \\
      \hline
      name     & type      & comment                   & implemented interface \\
      \hline
      TokenId  & string    & generated id of the token & erc721                \\
      Type     & string    & vaccine type (enum)       & vaccinationSlot       \\
      Date     & time.Time & vaccination date          & vaccinationSlot       \\
      Previous & string    & vaccine type (enum)       & vaccinationSlot       \\
      Owner    & string    & owner of the token        & erc721                \\
      Burned   & boolean   & is it used up             & erc721 optional       \\
      Approved & string    &                           & erc721                \\
      \hline
    \end{tabular}
    \caption{\href{https://pkg.go.dev/github.com/perryd01/vaccination-slot/chaincode\#VaccinationSlot}{Vaccination Slot token's data model.}}
  \end{table}
\end{center}


\begin{center}
  \begin{table}[!ht]
    \centering
    \begin{tabular}{| c | c |}
      \hline
      \multicolumn{2}{|c|}{Approval} \\
      \hline
      name     & type                \\
      \hline
      Owner    & string              \\
      Operator & string              \\
      Approved & bool                \\
      \hline
    \end{tabular}
    \caption{\gopkg{\#Approval}{Approval} object's data model.}
  \end{table}
\end{center}

\begin{center}
  \begin{table}[!ht]
    \centering
    \begin{tabular}{| c | c |}
      \hline
      \multicolumn{2}{|c|}{Transfer} \\
      \hline
      name    & type                 \\
      \hline
      From    & string               \\
      To      & string               \\
      TokenId & string               \\
      \hline
    \end{tabular}
    \caption{\gopkg{\#Transfer}{Transfer} object's data model.}
  \end{table}
\end{center}

\newpage
\subsection{API}
Inherited from \href{https://eips.ethereum.org/EIPS/eip-721}{ERC-721 specification}, function names may differ. For more specific developer documentation \href{https://pkg.go.dev/github.com/perryd01/vaccination-slot/chaincode#section-documentation}{see generated \emph{godoc}.}
\subsubsection{Callable functions}
\begin{itemize}
  \item \function{\gopkg{\#VaccinationContract.BalanceOf}{BalanceOf}}{owner string}{int}{Returns number of tokens in owner's wallet.}
  \item \function{\gopkg{\#VaccinationContract.OwnerOf}{OwnerOf}}{tokenId string}{string}{Returns owner of token.}
  \item \function{\gopkg{\#VaccinationContract.TransferFrom}{TransferFrom}}{from string, to string, tokenId string}{bool}{Transfering a token from wallet A to wallet B (if successful). }
  \item \function{\gopkg{\#VaccinationContract.Approve}{Approve}}{operator string, tokenId string}{bool}{ Change or reaffirm the approved address for an NFT. }
  \item \function{\gopkg{\#VaccinationContract.SetApprovalForAll}{SetApprovalForAll}}{operator string, approved bool}{bool}{ Enable or disable approval for a third party ("operator") to manage all of `msg.sender`'s assets.  }
  \item \function{\gopkg{\#VaccinationContract.GetApproved}{GetApproved}}{tokenId string}{string}{ Get the approved address for a single NFT }
  \item \function{\gopkg{\#VaccinationContract.IsApprovedForALl}{IsApprovedForAll}}{owner string, operator string}{bool}{ Query if an address is an authorized operator for another address.  }
  \item \function{\gopkg{\#VaccinationContract.ClientAccountId}{ClientAccountId}}{}{string}{ Returns clientAccountId string }
  \item \function{\gopkg{\#VaccinationContract.GetSlots}{GetSlots}}{owner string}{VaccinationSlot[ ]}{ Queries vaccination slots belonging to owner.}
  \item \function{\gopkg{\#VaccinationContract.IssueSlot}{IssueSlot}}{vaccine string, date string, patient string}{string}{ Create's a slot (if client is authorized) and transfers to specific patient (wallet). }
  \item \function{\gopkg{\#VaccinationContract.MakeOffer}{MakeOffer}}{mySlotUuid, recipient, recipientSlotUuid string}{offerUuid string}{ Create an offer. }
  \item \function{\gopkg{\#VaccinationContract.AcceptOffer}{AcceptOffer}}{offerUuid string}{}{ Accept an offer. }
  \item \function{\gopkg{\#VaccinationContract.ListOffers}{ListOffers}}{}{string}{ List available offers. }
  \item \function{\gopkg{\#VaccinationContract.DeleteOffer}{DeleteOffer}}{offerUuid string}{}{ List available offers. }
  \item \function{\gopkg{\#VaccinationContract.BurnToken}{BurnToken}}{slotUuid string}{}{  }
\end{itemize}
\subsubsection{Non-callable functions}
\begin{itemize}
  \item \function{readVaccinationSlot}{tokenId string}{VaccinationSlot}{Retrives a token by tokenId.}
  \item \function{vaccinationSlotExists}{tokenId string}{bool}{Returns a boolean whether the token exists or not.}
\end{itemize}


\subsection{Implemention details}
There are doctors and patients. The doctors are able to mint and burn Vaccination Slot tokens. The most important properties of a single token are: Type \emph{(of the vaccine)}, Date \emph{(when the token should be burned)}, Burned \emph{(is it used up)}, Previous \emph{(previous vaccine type the patient got)}.
Type can be any of the following: \emph{Alpha}, \emph{Bravo}, \emph{Charlie}, \emph{Delta}, \emph{Echo}. Date represents a single day. For a single day, all permutation can be minted by doctors, so two tokens can exist with the same type and date but different tokenIds and held by different patients.

A patient can hold only one non-burned token and unlimited number of burned. A patient cannot trade a burned token.
A patient can trade a valid token disregarding the previous burned token. \emph{If a patient's first vaccine was an Alpha one and got another Alpha token from the doctors, it is allowed to trade it for a Bravo token.}


\newpage
\subsection{Test cases}
Doctor wallets are created when the network starts running.

\testStruct{Successful token minting}{  \item create network
  \item create wallet A
  \item doctor mints token X and transfers to A}{Wallet A has valid token X.}

\testStruct{Unauthorized token minting}{  \item create network
  \item create wallet A
  \item create wallet B
  \item wallet A mints token X and transfers to B}{Minting unsuccessful, wallet A isn't authorized to mint tokens.}

\testStruct{Successful trading}{  \item create network
  \item create wallet A
  \item create wallet B
  \item doctor mints token X and transfers to A
  \item A transfers token X to wallet B}{Trading successful, wallet B has token X.}

\testStruct{Trading a burned token}{\item create network
  \item create wallet A
  \item create wallet B
  \item doctor mints token X and transfers to A
  \item wallet A uses token X, token get's burned by doctor
  \item A transfers token X to wallet B
}{Trading unsuccessful, wallet A cannot trade a burned token.}

\testStruct{Trading a token with invalid date}{
  \item create network
  \item create wallet A
  \item create wallet B
  \item doctor mints token X and transfers to A
  \item A transfers token X to wallet B
}{Trading unsuccessful, wallet A cannot trade a token with invalid date.}

\testStruct{Trading a nonexistent token}{
  \item create network
  \item create wallet A
  \item create wallet B
  \item A transfers token X to wallet B
}{Trading unsuccessful, wallet A cannot trade a nonexistent token.}
