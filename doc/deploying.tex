Your are going to need to set up: \emph{vscode, docker, golang}.
\subsection{Useful links for installing}
\begin{itemize}
    \item docker: \href{https://docs.docker.com/get-docker/}{https://docs.docker.com/get-docker/}
    \item vscode: \href{https://code.visualstudio.com/download}{https://code.visualstudio.com/download}
    \item golang: \href{https://go.dev/doc/install}{https://go.dev/doc/install}
\end{itemize}
\subsection{Follow these steps}
\begin{itemize}
    \item Open vscode
    \item Install IBM Blockchain Platform extension
    \item Setup Docker and make sure Docker daemon is runnning
    \item \begin{verbatim}$ go mod download\end{verbatim}
    \item \begin{verbatim}$ go run ./tools/network_setup/main.go\end{verbatim}
    \item add the network as a Microfab network using the default URL: \begin{verbatim}
        http://console.127-0-0-1.nip.io:8080
    \end{verbatim}
    \item do stuff as you would do on any other network
    \item Done.
\end{itemize}

\subsection{Network starter script}
Execute this command to start the network.
\begin{verbatim}$ go run ./tools/network_setup/main.go\end{verbatim}
\subsubsection{Reuse}
\begin{verbatim}
    -reuse
\end{verbatim}
Reuse existing docker container if possible, defualt false. It will be looking for an existing stopped container that has the same \hyperref[ref:ContainerName]{Container Name} as the one you are trying to start.

\subsubsection{Host Port}
\begin{verbatim}
    -hport=<port>
\end{verbatim}
Set host port for docker container, default 8080.

\subsubsection{Container Name}\label{ref:ContainerName}
\begin{verbatim}
    -cname=<name>
\end{verbatim}
Set name of docker container, default "vacc{\_}slot".